\documentclass[11pt]{beamer}
\usepackage[utf8]{inputenc}
\usepackage[T1]{fontenc}
\usetheme[compress]{Berlin}

\begin{document}
	\author{Valerian Mahdi Pratama (10117049)}
	\title{Kalimat Terbuka, Apa Itu?}
	\subtitle{MA2111 Pengantar Matematika}
	\institute[Prodi Matematika FMIPA ITB]{Program Studi Matematika \\Fakultas Matematika dan Ilmu Pengetahuan Alam \\Institut Teknologi Bandung}
	\date{}
	%Mengatur opacity dari teks transparan yang muncul saat \pause
	\setbeamercovered{transparent=10}
\begin{frame}[plain]
\maketitle
\end{frame}
\section{Pendahuluan}
\begin{frame}{Sebelum kita mulai..}
Mari kita ingat-ingat dulu, apa sih proposisi itu? Kalau variabel, itu apa?
\begin{block}{Proposisi}
	\pause
	\textbf{Definisi.} Kalimat deklaratif yang dapat bernilai benar atau salah, tetapi tidak keduanya. Kebenaran atau kesalahan dari kalimat tersebut disebut sebagai nilai kebenaran (\textit{truth value}) \cite{RM14}.
\end{block}
\begin{block}{Variabel}
	\pause
	\textbf{Definisi.} Simbol yang merepresentasikan suatu objek yang tak ditentukan yang bisa dipilih dari suatu himpunan $U$. Himpunan $U$ adalah himpunan semesta untuk suatu variabel yang anggotanya boleh dipilih untuk menggantikan variabel \cite{TS18}.
\end{block}
\end{frame}
\section{Kalimat Terbuka}
\begin{frame}{Kalimat Terbuka}
\pause
\begin{block}{Kalimat Terbuka}
	\textbf{Definisi.} Kalimat terbuka adalah suatu kalimat $P(x_1, x_2, ... , x_n)$ yang melibatkan variabel $x_1, x_2, ..., x_n$ yang kemudian mengakibatkan suatu kalimat tersebut dapat bernilai benar atau salah setelah variabel-variabel tersebut diberi nilai. Kalimat tersebut kemudian berubah menjadi pernyataan/proposisi. \cite{TS18}
\end{block}
\pause
Kalimat terbuka disebut juga sebagai predikat atau fungsi proposisi.
\end{frame}
\begin{frame}{Contoh}
Lihat beberapa kalimat terbuka berikut:\\
$P(x) :=$ Valerian semester ini mengambil mata kuliah $x$\\
\pause
$Q(x) := x^2 = -1$\\
\pause
$R(x,y,z) := x^2 + y^2 + z^2 = 1$\\
\pause
$H(A) :=$ $x = A$ adalah solusi dari $x^2 - x - 2= 0$
\pause\\

Sekarang mari periksa nilai kebenaran dari:\\
1. $P($"Pengantar Matematika"$)$\pause\\2. $Q(0)$\pause\\3. $H(2)$\pause\\4. $R(0,0,1)$\pause\\5. $R(1,1,-1)$
\end{frame}
\section{Himpunan Kebenaran}
\begin{frame}{Himpunan Kebenaran}
\begin{block}{\textit{Truth Set}}
	\pause
	\textbf{Definisi.} Himpunan kebenaran (\textit{truth set}) dari suatu kalimat terbuka adalah sekumpulan objek dalam himpunan universal yang dapat disubtitusikan ke variabel dalam kalimat terbuka dan menjadikan kalimat terbuka tersebut pernyataan bernilai benar. \cite{TS18}
\end{block}
\end{frame}
\begin{frame}{Himpunan Kebenaran - Contoh (1)}
Mari kita tinjau kembali contoh kalimat terbuka yang kita bahas sebelumnya.\\\pause
$P(x) :=$ Valerian semester ini mengambil mata kuliah $x$
\pause
\\
Untuk kalimat $P(x)$, himpunan universalnya adalah $M =$ \{Himpunan semua mata kuliah yang ada\} dan himpunan kebenarannya adalah $\{x \in M$ | $P(x)\}$, yaitu semua $x$ dalam M yang membuat $P(x)$ benar.
\pause
\\
Sekarang untuk kalimat $Q(x)$ := $x^2 = -1$. Untuk himpunan universal $\mathbb{R}$ himpunan kebenarannya adalah $\emptyset$. Jika kita ganti himpunan universalnya menjadi $\mathbb{C}$ himpunan kebenarannya menjadi $\{-i,i\}$.
\end{frame}
\begin{frame}{Himpunan Kebenaran - Contoh (2)}
Tinjau kalimat terbuka berikut
$$P(x) := x^2 - x + 1\;habis\;dibagi\;2$$
Untuk $x \in \mathbb{Z}^+$, apakah himpunan kebenarannya?
\end{frame}
\begin{frame}{Himpunan Kebenaran - Contoh (2, cont.)}
Kalimat tersebut sama saja dengan mencari $x$ yang memenuhi $x^2 - x + 1 \equiv 0\;(mod\;2)$. Karena $x$ diharuskan bilangan bulat positif maka himpunan universal kita adalah $\mathbb{Z}^+$.\\\pause Dengan sifat modulo, kita tahu bahwa $$(x^2 - x + 1)\;mod\;2 = ((x\;mod\;2)^2-(x\;mod\;2) + 1)\;mod 2$$\pause Dengan kata lain, agar bernilai nol maka haruslah $(x\;mod\;2)^2-(x\;mod\;2) = -1$. Karena $-1 \equiv 1 (mod\;2)$, maka haruslah $(x\;mod\;2)^2-(x\;mod\;2) = 1$.\pause Untuk $x \in \mathbb{Z}^+$ tidak ada $x$ yang memenuhi sehingga \textit{truth set} dari kalimat tersebut adalah $\emptyset$.
\end{frame}
\begin{frame}{Himpunan Kebenaran - Contoh (2, cont.)}
Bagaimana kalau kalimatnya diubah menjadi $P'(x) := x^2 - x + 1$ bernilai ganjil?
\pause\\

Sebelumnya kita telah tunjukkan bahwa tidak ada $x$ bulat positif yang menjadikan polinom tersebut genap. Oleh karena itu \textit{truth set} dari kalimat $P'(x)$ adalah $\mathbb{Z}^+ - \emptyset = \mathbb{Z}^+$.
\pause\\

Secara lebih umum, jika kita punya himpunan universal $H$ dari suatu kalimat terbuka dan himpunan kebenarannya T, jika kita ingin mencari kebalikannya (himpunan ketika nilainya menjadi salah), maka kita punya himpunan $
H-T$.
\end{frame}
\begin{frame}{Himpunan Kebenaran}
Beberapa metode untuk mencari himpunan kebenaran dari suatu kalimat terbuka:
\begin{itemize}
	\item Menyocokkan dengan himpunan universal
	\item Melakukan operasi matematika untuk menyelesaikan persamaan
	\item \textit{Prior math knowledge}
\end{itemize}
\end{frame}
\section{Referensi}
\begin{frame}{Referensi}
%Style referensi dengan panduan gaya APA
\bibliographystyle{apalike}
%Refer ke file .bib (harus ada di direktori yang sama)
\bibliography{openstate}
\end{frame}
\end{document}